\documentclass{article}

\usepackage[utf8]{inputenc}
\usepackage{float}
\usepackage{booktabs}
\usepackage{titling}
\usepackage{listings}
\usepackage{graphicx}
\usepackage{caption}
\usepackage{subcaption}
\usepackage[margin=0.5in]{geometry}
\usepackage{hyperref}



\title{Predicting Runs}
\author{Derek Owens-Oas, Fan Bu, Federico Ferrari, Megan Robertson}
\date{September 24, 2017}

\begin{document}

\maketitle

\section{Introduction}

We chose to investigate the fourth prompt during the NBA Hackathon - create a model to predict exciting runs during games. One way to define a run is when a team scores on multiple consecutive possessions without the opponent scoring any points in between. This definition captures many runs, but it does not capture the scenario in which a team makes three-pointers on three consecutive possessions, and the other team scores a quiet lay-up in between. We propose a novel and flexible definition of runs which captures scenarios like this. \newline

With runs well defined, we set out with a goal to learn which covariates, or features of the current gameplay, correlate strongly with runs. A few specific hypotheses motivate our analysis. We think there are scenarios in which a streak is more likely. A motivating halftime speech, a large skill differential between the offense?s and defense?s lineups, or an opponent with a propensity to turnover the ball may increase the likelihood of a streak. On the other hand, some features may decrease the likelihood of a run. Playing the second game of a back to back is one example.

\section{Defining an "exciting" run}

In order to make the project approachable during a 24-hour time period, a simplified definition of run was used to classify every possession in the 2016-2017 season. The algorithm below was used to create the labels for the possession log data. 

\begin{figure}[h]
\begin{center}
\includegraphics[width=100mm]{run_alg.png}
\caption{Algorithm for Defining Run}
\end{center}
\end{figure}

PLAINER ENGLISH FOR THE NON-STATISTICIANS (THE LAMESIANS, GET IT, NOT BAYESIAN!!!)

- percentage of possessions that are part of a run and those that are not?

\section{Feature Generation}

Multiple variables were added to the possession log data that we thought might be indicative of whether a run was occurring at a certain possession in the game.  Two components that define a run are the pace of the game as well as the points being scored. Typically runs are the result of many rapid plays. Therefore, variables were added for the number of shots taken during a possession as well as the number of rebounds. More shots and more rebounds would be indicative of a team getting many offensive boards and having to work for their basket. The type of shot is also very important to consider when exploring runs. Discussions with our coach and other league mentors led us to add an indicator variable defining whether a shot was a "good" shot. This was arbitrarily defined as any shot that was within six feet of the hoop. This is roughly the areas in and around the key closest to the hoop. \newline

In addition to capturing the variability of runs through the above variables, we also wanted to capture the skill level of the players on the court at the time of the run. Using the play by play data, it was possible to determine the players on the court at the time of each possession. From this, we were able to use external data \footnote{"NBA All Star", \url{http://www.nba-allstar.com/players/lists/players-by-draft-pick.html, 9/23/17}} to determine the number of all stars on the court at the time. This is a very basic way to evaluate skill, but other methods could be used to define quality players. However, given the brief work period we only incorporated the all-star information for the time being. 

\section{Modeling}

We approached predicting exciting runs as a classification problem, each possession in a game can either be classified as being part of a run or not. 

\begin{figure}[h]
\centering
\begin{subfigure}{.5\textwidth}
  \centering
  \includegraphics[width=\linewidth]{Net_Point_21601204.jpeg}
  \label{fig:sub1}
\end{subfigure}%
\begin{subfigure}{.5\textwidth}
  \centering
  \includegraphics[width=\linewidth]{Probability_Plot.jpeg}
  \label{fig:sub2}
\end{subfigure}
\caption{Left - Model 1 Coefficients, Right - Model 2 Coefficients}
\label{fig:test}
\end{figure}




\section{Results}

\begin{figure}[h]
\begin{center}
\includegraphics[width=175mm]{results.pdf}
\caption{Modeling Results. Model 1 - Odds of going on a run, Model 2 - Odds of allowing a run}
\end{center}
\end{figure}


\begin{figure}[h]
\centering
\begin{subfigure}{.5\textwidth}
  \centering
  \includegraphics[width=\linewidth]{odds_run_for.png}
  \label{fig:sub1}
\end{subfigure}%
\begin{subfigure}{.5\textwidth}
  \centering
  \includegraphics[width=\linewidth]{odds_run_against.png}
  \label{fig:sub2}
\end{subfigure}
\caption{Left - Model 1 Coefficients, Right - Model 2 Coefficients}
\label{fig:test}
\end{figure}

\section{Next Steps}

This project was an attempt to answer a very complex question in a sort period of time. There are a few routes we would consider if we had more time to continue the project. To begin with, there is more data that will most likely capture the variability of whether a possession could be classified as a run. At the moment, the skill of the players is captured by the number of all stars on the court during a possession. However, analyzing different combinations of players, particularly when there are starting players vs. bench players, etc, would be interesting to examine in regards to runs. \newline

Another next step would be to examine different types of classification models. There are many different types of machine learning algorithms that could be implemented in such a project. A more abstract model, such as a decision tree or a support vector machine might capture some of the complicated relationships that are present in the data. 
 

\end{document}